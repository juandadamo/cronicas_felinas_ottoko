%\newpage
%\begin{tikzpicture}[remember picture, overlay]
%	\node [inner sep=0pt, minimum width=\paperwidth, minimum height=\paperheight,opacity=1,color=Apricot] at (current page.center) {\includegraphics[width=\paperwidth,height=\paperheight,angle=0]{paper18}};
%
%\end{tikzpicture}
%CUANDO ME VEN, A PESAR DE MI PEQUEÑO TAMAÑO, PUEDEN PENSAR QUE SOY UNA PANTERA NEGRA. S



\newpage
\begin{tikzpicture}[remember picture, overlay]
	\node [inner sep=0pt, minimum width=\paperwidth, minimum height=\paperheight,opacity=.3,color=ForestGreen,fill=ForestGreen] at (current page.center){};
\end{tikzpicture}
\begin{wrapfigure} [7]{l}{.2\textwidth}\vspace{-1.2cm}\hspace{-1.5cm}
	\begin{tikzpicture}
		\node[xscale=-1,yshift=2cm] () {\includegraphics[width=.25\textwidth]{gato_azul.png}};
	\end{tikzpicture}
	
	\end{wrapfigure}	
ANOCHE SOÑE CON UN GATO AZUL. NUNCA HABÍA VISTO UNO DE ESE COLOR, TAL VEZ CUANDO MIRÉ LA PELÍCULA DE ALICIA EN EL PAÍS DE LAS MARAVILLAS, AQUEL GATO SONRIENTE DE CHESHIRE. PERO NO, EL MÍO NO TENÌA UNA SONRISA DE DIENTES INTERMINABLES, NI JUGABA CON MI CREDULIDAD DESAPARECIENDO Y GIRANDO COMO UN TROMPO. MI GATO AZUL ERA COMO UN PEQUEÑO FANTASMA SERVICIAL QUE PODÍA AYUDARME CON MIS DIBUJOS.

PARA ELLO TENÍA ABIERTOS BIEN GRANDES LOS OJOS, PUES PARA ILUSTRAR BIEN HAY QUE SABER PRIMERO MIRAR. ASÍ COMO PARA HABLAR LO HACEMOS DESPUÉS DE ESCUCHAR Y ESCRIBIR, LUEGO DE LEER. 

MI AMIGO PORTABA UN PLUMÍN SINGULAR.
\newpage
\begin{tikzpicture}[remember picture, overlay]
	\node [inner sep=0pt, minimum width=\paperwidth, minimum height=\paperheight,opacity=.4,color=Apricot] at (current page.center) {\includegraphics[width=\paperwidth,height=\paperheight,angle=0]{paper27}};
\end{tikzpicture}
\begin{wrapfigure} [9]{l}{.45\textwidth}\vspace{-1.2cm}%\hspace{-1.5cm}
	\begin{tikzpicture}
		\node[xscale=1,yshift=0cm] () {\includegraphics[width=.45\textwidth]{remolino1.png}};
	\end{tikzpicture}
\end{wrapfigure}
CUANDO TRAZÓ CON SU PLUMÍN LA FORMA QUE PUEDEN VER, LO PRIMERO QUE PENSÉ ES QUE IBA A SER UN AUTORETRATO QUE COMENZABA CON SU COLA. LUEGO PENSÉ EXTRAÑADO EN LAS PROPORCIONES Y ME FIGURÉ QUE SE TRATARÍA DE UNA OLA, ASÍ COMO SE LUCEN JUSTO ANTES DE ROMPER EN LA PLAYA.

LUEGO RECORDÉ LO QUE HABÍA LEÍDO EN EL PRINCIPITO, QUE NADA NECESARIAMENTE ES LO QUE PARECE A LA PRIMERA IMPRESIÓN Y QUE UN SOMBRERO PUEDE EN REALIDAD SER UN ELEFANTE TRAGADO POR UNA GIGANTESCA BOA.
\newpage
\begin{tikzpicture}[remember picture, overlay]
	\node [inner sep=0pt, minimum width=\paperwidth, minimum height=\paperheight,opacity=.3,color=Apricot] at (current page.center) {\includegraphics[width=\paperwidth,height=\paperheight,angle=0]{paper27}};
\end{tikzpicture}
\begin{wrapfigure}[8]{l}{.4\textwidth}\vspace{-1.2cm}%\hspace{-1.5cm}
	\begin{tikzpicture}
		\node[xscale=1,yshift=0cm] () {\includegraphics[width=.43\textwidth]{remolino3.png}};
	\end{tikzpicture}
\end{wrapfigure}
ASÍ QUE REFLEXIONANDO BIEN, LA PINTURA PUEDE ACOMODARSE COMO UN REMOLINO, GIRANDO COMO UNA ESPIRAL. Y COMO ESTA, HAY FORMAS QUE SE PRESENTAN A TODA ESCALA, DESDE LAS DISTANCIAS ASTRONÓMICAS DE LA GALAXIA, HASTA LAS PEQUEÑAS CADENAS EN LOS NÚCLEOS DE LAS CÉLULAS.
ME GUSTA TAMBIÉN PENSAR EN QUE HAY UNA ESPECIE DE ATRACTOR QUE CONDUCE AQUELLO QUE ESTÁ LEJOS HACIA UN CENTRO. Y QUE AÚN CUANDO ALGÚN CAMINO PUEDA PARECER LEJANO, PRONTO ENCONTRAREMOS OTRO QUE NOS PUEDE CONDUCIR DE NUEVO AL MISMO CENTRO. EN ESTAS COSAS Y MÁS PENSABA MIRANDO LOS TRAZOS.

\newpage
\begin{tikzpicture}[remember picture, overlay]
	\node [inner sep=0pt, minimum width=\paperwidth, minimum height=\paperheight,opacity=.3,color=Apricot] at (current page.center) {\includegraphics[width=\paperwidth,height=\paperheight,angle=0]{paper27}};
\end{tikzpicture}
LOS TRAZOS TAMBIÉN SE DIBUJAN SI CAMINAMOS EN UNA CALESITA. AUNQUE EN CASA CONOCÍ PRIMERO LA SILLA GIRATORIA DE MI PAPÁ, VI QUE EN VARIAS PLAZAS DE BARRIO HAY MUY BELLAS CALESITAS QUE GIRAN CON EL MISMO PRINCIPIO.